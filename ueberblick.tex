\documentclass[
  a4paper,
  11pt,
]{article}

\usepackage[utf8]{inputenc}
\usepackage[cm, headings]{fullpage}
\usepackage[ngerman]{babel}
\usepackage{amsmath}
\usepackage{amssymb}
\renewcommand{\O}{\mathcal{O}}

\usepackage{color}
\definecolor{mygray}{rgb}{0.5,0.5,0.5}
\usepackage{listings}
\usepackage{tikz}
\usepackage{pgfplots}

% Für Zeilenumbrüche ohne Indentations
\setlength{\parindent}{0pt}

\lstset{%
  basicstyle=\ttfamily,
  keywordstyle=\color{blue},
  commentstyle=\color{mygray},
  language=C,
  showstringspaces=false,
}

% coole Kopf- und Fußzeilen:
\usepackage{fancyhdr}
% Seitenstil ist natürlich fancy:
\pagestyle{fancy}
% alle Felder löschen:
\fancyhf{}

\fancyhead[L]{%
  Seminar: Public-Key-Kryptographie
}
\fancyhead[R]{%
}
%\fancyfoot[L]{}
\fancyfoot[C]{\thepage}

\newcommand{\N}{\mathbb{N}}
\newcommand{\ggT}{\text{ggT}}

\title{}

\author{}

\begin{document}

\section*{Themen}
\label{sec:Themen}

\subsection*{Theorie}
\label{sub:Theorie}

\begin{itemize}
  \item Division mit Rest, Modulo

  \item Gruppentheorie

    Was ist eine Gruppe?

      Eine Gruppe ist ein Tupel $(G, *)$ mit einer Menge $G$ und einer
      Verknüpfung $*: G \times G \to G$ bzw. $(a, b) \mapsto a * b$, welche
      folgende Axiome erfüllt:
      \begin{enumerate}
        \item Assoziativgesetz: für alle $a,b,c \in G$ gilt: $(a * b) * c = a
          * (b * c)$.
        \item neutrales Element: es gibt ein Element $e \in G$, so dass für
          alle $a \in G$ gilt: $a * e = e * a = a$.
        \item inverse Elemente: für alle $a \in G$ existiert ein inverses
          Element $a^{-1}$, sodass gilt $a * a^{-1} = a^{-1} * a = e$
      \end{enumerate}

  \item Beispiel: multiplikative Gruppe modulo $p$

  \item Erzeuger, zyklische Gruppe
\end{itemize}

\subsection*{Praxis}
\label{sub:Praxis}

\begin{itemize}
  \item DH-Schlüsselaustausch
  \item Verschlüsselung mittels diskretem Logarithmus (was ist der öffentliche
    und der private Schlüssel?)
  \item Gebräuchlichkeit?
\end{itemize}


\end{document}
