\documentclass[
  a4paper,
  11pt,
]{article}

\usepackage[utf8]{inputenc}
\usepackage[cm, headings]{fullpage}
\usepackage[ngerman]{babel}
\usepackage{amsmath}
\usepackage{amssymb}
\renewcommand{\O}{\mathcal{O}}

\usepackage{color}
\definecolor{mygray}{rgb}{0.5,0.5,0.5}
\usepackage{listings}
\usepackage{tikz}
\usepackage{pgfplots}
\usepackage{booktabs}

% Für Zeilenumbrüche ohne Indentations
\setlength{\parindent}{0pt}

\lstset{%
  basicstyle=\ttfamily,
  keywordstyle=\color{blue},
  commentstyle=\color{mygray},
  language=C,
  showstringspaces=false,
}

% coole Kopf- und Fußzeilen:
\usepackage{fancyhdr}
% Seitenstil ist natürlich fancy:
\pagestyle{fancy}
% alle Felder löschen:
\fancyhf{}

\fancyhead[L]{%
  Seminar: Public-Key-Kryptographie
}
\fancyhead[R]{%
}
%\fancyfoot[L]{}
\fancyfoot[C]{\thepage}

\newcommand{\N}{\mathbb{N}}
\newcommand{\Z}{\mathbb{Z}}
\newcommand{\ggT}{\text{ggT}}

\title{}

\author{}

\begin{document}

\section*{Themen}
\label{sec:Themen}

\subsection*{Theorie}
\label{sub:Theorie}

\begin{itemize}
  \item Division mit Rest, Modulo

  \item Gruppentheorie

    Was ist eine Gruppe?

      Eine Gruppe ist ein Tupel $(G, *)$ mit einer Menge $G$ und einer
      Verknüpfung $*: G \times G \to G$ bzw. $(a, b) \mapsto a * b$, welche
      folgende Axiome erfüllt:
      \begin{enumerate}
        \item Assoziativgesetz: für alle $a,b,c \in G$ gilt: $(a * b) * c = a
          * (b * c)$.
        \item neutrales Element: es gibt ein Element $e \in G$, so dass für
          alle $a \in G$ gilt: $a * e = e * a = a$.
        \item inverse Elemente: für alle $a \in G$ existiert ein inverses
          Element $a^{-1}$, sodass gilt $a * a^{-1} = a^{-1} * a = e$
      \end{enumerate}

  \item Beispiel: multiplikative Gruppe modulo $n$

    Wir betrachten $\Z/5\Z$ mit der Multiplikation modulo $5$. Dann ergibt sich
    folgende Multiplikationstabelle:
    \begin{center}
      \begin{tabular}{r|cccc}
        $\cdot$ & $1$ & $2$ & $3$ & $4$\\\hline
          $1$   & $1$ & $2$ & $3$ & $4$\\
          $2$   & $2$ & $4$ & $1$ & $3$\\
          $3$   & $3$ & $1$ & $4$ & $2$\\
          $4$   & $4$ & $3$ & $2$ & $1$
      \end{tabular}
    \end{center}
    Die $0$ wird aus $\Z/5\Z$ ausgeschlossen, da für die $0$ kein inverses
    Element existiert und damit ein Gruppenaxiom verletzt würde.

    Betrachten wir ein Negativbeispiel: $\Z/6\Z$ mit der Multiplikation modulo
    $6$. Dies ist keine Gruppe, da für $2$ kein inverses Element existiert. Dies
    kann man an der Multiplikationstabelle erkennen:
    \begin{center}
      \begin{tabular}{r|ccccc}
        $\cdot$ & $1$ & $2$ & $3$ & $4$ & $5$\\\hline
        $1$     & $1$ & $2$ & $3$ & $4$ & $5$\\
        $2$     & $2$ & $4$ & $0$ & $2$ & $4$\\
        $3$     & $3$ & $0$ & $3$ & $0$ & $3$\\
        $4$     & $4$ & $2$ & $0$ & $4$ & $2$\\
        $5$     & $5$ & $4$ & $3$ & $2$ & $1$
      \end{tabular}
    \end{center}

    Wir schreiben $\Z_n^*$ für die multiplikative Gruppe modulo $n$.

  \item Erzeuger, zyklische Gruppe

    Sei $(G, \cdot)$ eine Gruppe. Wenn es ein Element $a \in G$ gibt, dessen
    Potenzen $a^n$ für $n \in \Z$ alle Elemente aus $G$ erzeugt, dann nennt man
    $a$ einen Erzeuger und $(G, \cdot)$ eine \emph{zyklische} Gruppe.

    Wir schreiben $\left\langle a \right\rangle := \left\{ a^n \mid n \in \Z
    \right\}$, um $a$ als erzeugendes Element zu kennzeichnen.

    In einer multiplikativen Gruppe modulo $n$ heißen die Erzeuger
    \emph{Primitivwurzeln} modulo $n$.

  \item Beispiel: Erzeuger

    In $\Z_5^*$ sind $2$ und $3$ Erzeuger, wie man durch Ausprobieren nachprüfen
    kann:
    \begin{center}
      \begin{tabular}{r|rrrr}
        $n$   & $1$ & $2$ & $3$ & $4$\\\hline
        $2^n$ & $2$ & $4$ & $1$ & $3$\\
        $3^n$ & $3$ & $4$ & $2$ & $1$
      \end{tabular}
    \end{center}
    Die $4$ hingegen ist kein Erzeuger:
    \begin{center}
      \begin{tabular}{r|rrrr}
        $n$   & $1$ & $2$ & $3$ & $4$\\\hline
        $4^n$ & $4$ & $1$ & $4$ & $1$
      \end{tabular}
    \end{center}

\end{itemize}

\subsection*{Praxis}
\label{sub:Praxis}

\begin{itemize}
  \item DH-Schlüsselaustausch
  \item Verschlüsselung mittels diskretem Logarithmus (was ist der öffentliche
    und der private Schlüssel?)
  \item Gebräuchlichkeit?
\end{itemize}


\end{document}
