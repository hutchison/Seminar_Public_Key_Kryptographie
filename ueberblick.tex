\documentclass[
  a4paper,
  11pt,
]{article}

\usepackage[utf8]{inputenc}
\usepackage[cm, headings]{fullpage}
\usepackage[ngerman]{babel}
\usepackage{amsmath}
\usepackage{amssymb}
\renewcommand{\O}{\mathcal{O}}

\usepackage{color}
\definecolor{mygray}{rgb}{0.5,0.5,0.5}
\usepackage{listings}
\usepackage{tikz}
\usepackage{pgfplots}

% Für Zeilenumbrüche ohne Indentations
\setlength{\parindent}{0pt}

\lstset{%
  basicstyle=\ttfamily,
  keywordstyle=\color{blue},
  commentstyle=\color{mygray},
  language=C,
  showstringspaces=false,
}

% coole Kopf- und Fußzeilen:
\usepackage{fancyhdr}
% Seitenstil ist natürlich fancy:
\pagestyle{fancy}
% alle Felder löschen:
\fancyhf{}

\fancyhead[L]{%
  Seminar: Public-Key-Kryptographie
}
\fancyhead[R]{%
}
%\fancyfoot[L]{}
\fancyfoot[C]{\thepage}

\title{}

\author{}

\begin{document}

\section*{Themen}
\label{sec:Themen}

\subsection*{Theorie}
\label{sub:Theorie}

\begin{itemize}
  \item Gruppentheorie
  \item Multiplikative Gruppe modulo $p$
  \item Eulersche $\phi$-Funktion
  \item primitive Wurzeln und wie man sie findet
\end{itemize}

\subsection*{Praxis}
\label{sub:Praxis}

\begin{itemize}
  \item $a^b \mod n$ schnell berechnen, Square and Multiply
  \item Berechnung vom diskreten Logarithmus (einfach und schneller – siehe
    Babystep-Giantstep-Algorithmus)
  \item DH-Schlüsselaustausch
  \item Verschlüsselung mittels diskretem Logarithmus (was ist der öffentliche
    und der private Schlüssel?)
  \item Gebräuchlichkeit?
\end{itemize}


\end{document}
