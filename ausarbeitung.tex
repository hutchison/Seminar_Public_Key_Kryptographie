\documentclass[
  a4paper,
  11pt,
]{scrartcl}

\usepackage[utf8]{inputenc}
\usepackage[cm, headings]{fullpage}
\usepackage[ngerman]{babel}
\usepackage{amsmath}
\usepackage{amssymb}
\usepackage{amsthm}

\theoremstyle{plain}
\newtheorem{satz}{Satz}
\theoremstyle{definition}
\newtheorem{definition}{Definition}
\theoremstyle{remark}
\newtheorem{beispiel}{Beispiel}

\usepackage{color}
\definecolor{mygray}{rgb}{0.5,0.5,0.5}
\usepackage{listings}
\usepackage{tikz}
\usepackage{pgfplots}
\usepackage{booktabs}

% Für Zeilenumbrüche ohne Indentations
\setlength{\parindent}{0pt}

\lstset{%
  basicstyle=\ttfamily,
  keywordstyle=\color{blue},
  commentstyle=\color{mygray},
  language=python,
  showstringspaces=false,
}

% coole Kopf- und Fußzeilen:
\usepackage{fancyhdr}
% Seitenstil ist natürlich fancy:
\pagestyle{fancy}
% alle Felder löschen:
\fancyhf{}

\fancyhead[L]{%
  Seminar: Public-Key-Kryptographie
}
\fancyhead[R]{%
}
%\fancyfoot[L]{}
\fancyfoot[C]{\thepage}

\newcommand{\N}{\mathbb{N}}
\newcommand{\Z}{\mathbb{Z}}
\newcommand{\ggT}{\text{ggT}}

\title{Diskrete Logarithmen}

\subtitle{Seminar: Public-Key-Kryptographie}

\author{%
  Martin Darmüntzel \and Hannes Kleinwort
}

\begin{document}

\maketitle

\section{Mathematische Grundlagen}
\label{sec:mathematische_grundlagen}

\begin{definition}[Kongruenz]
  Zwei ganze Zahlen $a, b$ sind \emph{kongruent} bezüglich eines Moduls
  $m \in \Z$, wenn sie bei der Division durch $m$ denselben Rest haben.
  Alternativ kann man Kongruenz auch auf den Teilbarkeitsbegriff zurückführen:
  \begin{align*}
    a \equiv b \mod m \Leftrightarrow m \mid (a - b)
  \end{align*}
\end{definition}

\begin{definition}[Gruppe]
  Eine \emph{Gruppe} ist ein Tupel $(G, \circ)$ mit einer Menge $G$ und einer
  Verknüpfung $\circ: G \times G \to G$ bzw. $(a, b) \mapsto a \circ b$, welche
  folgende Axiome erfüllt:
  \begin{enumerate}
    \item Assoziativgesetz: für alle $a, b, c \in G$ gilt:
      $(a \circ b) \circ c = a \circ (b \circ c)$.
    \item neutrales Element: es gibt ein Element $e \in G$, so dass für
      alle $a \in G$ gilt: $a \circ e = e \circ a = a$.
    \item inverse Elemente: für alle $a \in G$ existiert ein inverses
      Element $a^{-1}$, sodass gilt $a \circ a^{-1} = a^{-1} \circ a = e$
  \end{enumerate}
\end{definition}

\begin{definition}[Erzeuger, zyklische Gruppe]
  Sei $(G, \cdot)$ eine Gruppe. Wenn es ein Element $a \in G$ gibt, dessen
  Potenzen $a^n$ für $n \in \Z$ alle Elemente aus $G$ erzeugt, dann nennt man
  $a$ einen Erzeuger und $(G, \cdot)$ eine \emph{zyklische} Gruppe.

  Wir schreiben
  $\left\langle a \right\rangle := \left\{ a^n \mid n \in \Z \right\}$,
  um $a$ als erzeugendes Element zu kennzeichnen.

  In einer multiplikativen Gruppe modulo $p$ heißen die Erzeuger auch
  \emph{Primitivwurzeln} modulo $p$.
\end{definition}

\begin{definition}[Diskreter Logarithmus]
  Es sei $(G, \circ)$ eine endliche zyklische Gruppe mit Erzeuger $g$ der
  Ordnung $n$. Dann ist die diskrete Exponentiation
  \begin{align*}
    \exp_g: \quad \Z/n\Z & \to G\\
    x \mapsto g^x
  \end{align*}
  ein Gruppenisomorphismus. Die dazugehörige Umkehrfunktion
  \begin{align*}
    \log_g: \quad G & \to \Z/n\Z\\
    x & \mapsto \log_g x
  \end{align*}
  heißt \emph{diskreter Logarithmus} zur Basis $g$ und ist ebenfalls ein
  Gruppenisomorphismus.
\end{definition}

\begin{definition}[Diskretes Logarithmusproblem]
  Sei $p$ eine Primzahl, $g$ ein Erzeuger von $\Z_p^*$ und $y \in \Z_p^*$.

  Finde ein $x$ mit $1 \leq x \leq p-1$, so dass $g^x \equiv y \mod p$ gilt.
\end{definition}

\section{Anwendung: Ver- und Entschlüsselung}
\label{sec:anwendung_ver_und_entschlusselung}

\section{Berechnung von diskreten Logarithmen}
\label{sec:berechnung_von_diskreten_logarithmen}




\end{document}
