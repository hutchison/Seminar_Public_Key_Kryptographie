\documentclass[
  a4paper,
  11pt,
]{article}

\usepackage[utf8]{inputenc}
\usepackage[cm, headings]{fullpage}
\usepackage[ngerman]{babel}
\usepackage{amsmath}
\usepackage{amssymb}
\renewcommand{\O}{\mathcal{O}}

\usepackage{color}
\definecolor{mygray}{rgb}{0.5,0.5,0.5}
\usepackage{listings}
\usepackage{tikz}
\usepackage{pgfplots}
\usepackage{booktabs}

% Für Zeilenumbrüche ohne Indentations
\setlength{\parindent}{0pt}

\lstset{%
  basicstyle=\ttfamily,
  keywordstyle=\color{blue},
  commentstyle=\color{mygray},
  language=C,
  showstringspaces=false,
}

% coole Kopf- und Fußzeilen:
\usepackage{fancyhdr}
% Seitenstil ist natürlich fancy:
\pagestyle{fancy}
% alle Felder löschen:
\fancyhf{}

\fancyhead[L]{%
  Seminar: Public-Key-Kryptographie
}
\fancyhead[R]{%
  Beispiele
}
%\fancyfoot[L]{}
\fancyfoot[C]{\thepage}

\newcommand{\N}{\mathbb{N}}
\newcommand{\Q}{\mathbb{Q}}
\newcommand{\Z}{\mathbb{Z}}
\newcommand{\ggT}{\text{ggT}}

\title{}

\author{}

\begin{document}

\begin{itemize}
  \item Kongruenz:
    \begin{align*}
      18 & \equiv 3 \mod 15\\
      -10 & \equiv 20 \equiv 5 \mod 15\\
      16 & \not\equiv 2 \mod 15
    \end{align*}

  \item Gruppe:

    Sind das hier Gruppen?
    \begin{itemize}
      \item $(\Z, +)$
      \item $(\Q \setminus \{ 0 \}, \cdot)$
      \item $(\Z, \cdot)$
    \end{itemize}

  \item multiplikative Gruppen modulo $p$:

  Wir betrachten $\Z_5^*$. Dann ergibt sich folgende Multiplikationstabelle:
  \begin{center}
    \begin{tabular}{r|cccc}
      $\cdot$ & $1$ & $2$ & $3$ & $4$\\\hline
        $1$   & $1$ & $2$ & $3$ & $4$\\
        $2$   & $2$ & $4$ & $1$ & $3$\\
        $3$   & $3$ & $1$ & $4$ & $2$\\
        $4$   & $4$ & $3$ & $2$ & $1$
    \end{tabular}
  \end{center}

  Negativbeispiel: $\Z_4^*$. Dies ist keine Gruppe, da für $2$ kein inverses
    Element existiert. Dies kann man an der Multiplikationstabelle erkennen:
  \begin{center}
    \begin{tabular}{r|ccccc}
      $\cdot$ & $1$ & $2$ & $3$\\\hline
      $1$     & $1$ & $2$ & $3$\\
      $2$     & $2$ & $0$ & $2$\\
      $3$     & $3$ & $2$ & $1$
    \end{tabular}
  \end{center}

  \item Erzeuger:

    In $\Z_5^*$ sind $2$ und $3$ Erzeuger, wie man durch Ausprobieren nachprüfen
    kann:
    \begin{center}
      \begin{tabular}{r|rrrr}
        $n$   & $1$ & $2$ & $3$ & $4$\\\hline
        $2^n$ & $2$ & $4$ & $3$ & $1$\\
        $3^n$ & $3$ & $4$ & $2$ & $1$
      \end{tabular}
    \end{center}
    Die $4$ hingegen ist kein Erzeuger:
    \begin{center}
      \begin{tabular}{r|rrrr}
        $n$   & $1$ & $2$ & $3$ & $4$\\\hline
        $4^n$ & $4$ & $1$ & $4$ & $1$
      \end{tabular}
    \end{center}

  \item diskrete Exponentiation:

    Wir wählen $\Z_{11}^*$ als Gruppe und $7$ als Erzeuger. Dann erhalten wir
    für $\exp_7$ die folgende Wertetabelle:
    \begin{center}
      \begin{tabular}{r|rrrrrrrrrr}
        $x$ & $1$ & $2$ & $3$ & $4$ & $5$ & $6$ & $7$ & $8$ & $9$ & $10$\\\hline
        $7^x$ & $7$ & $5$ & $2$ & $3$ & $10$ & $4$ & $6$ & $9$ & $8$ & $1$
      \end{tabular}
    \end{center}

    Publikumsfrage: gegeben $\Z_{47}^*$ als Gruppe und $5$ als Erzeuger. Was ist
    $5^{11} \equiv y \mod 47$? Antwort: $y = 13$. Das ist einfach.

  \item diskreter Logarithmus:

    Wir wählen wieder $\Z_{11}^*$ als Gruppe und $7$ als Erzeuger. Dann erhalten
    wir für $\log_7$ die folgende Wertetabelle:
    \begin{center}
      \begin{tabular}{r|rrrrrrrrrr}
        $a$ & $1$ & $2$ & $3$ & $4$ & $5$ & $6$ & $7$ & $8$ & $9$ & $10$\\\hline
        $\log_7 a$ & $10$ & $3$ & $4$ & $6$ & $2$ & $7$ & $1$ & $9$ & $8$ & $5$
      \end{tabular}
    \end{center}

    Publikumsfrage: gegeben $\Z_{47}^*$ als Gruppe und $5$ als Erzeuger. Was ist
    $5^x \equiv 41 \mod 47$? Antwort: $y = 13$. Das ist schwierig.

  \item Diffie-Hellman-Schlüsselaustausch:

    Alice:
    \begin{itemize}
      \item wähle Primzahl $p = 17$
      \item wähle $g = 5$ als Erzeuger
      \item wähle geheimen Schlüssel $a = 10$
      \item berechne öffentlichen Schlüssel
        \begin{align*}
          A & \equiv g^a \mod p\\
            & \equiv 5^{10} \mod 17\\
            & \equiv 9
        \end{align*}
      \item sende $(p, g, A) = (17, 5, 9)$ an Bob
    \end{itemize}

    Bob:
    \begin{itemize}
      \item wähle geheimen Schlüssel $b = 7$
      \item berechne öffentlichen Schlüssel
        \begin{align*}
          B & \equiv g^b \mod p\\
            & \equiv 5^7 \mod 17\\
            & \equiv 10
        \end{align*}
      \item berechne geheimen gemeinsamen Schlüssel
        \begin{align*}
          K & \equiv A^b \mod p\\
            & \equiv 9^7 \mod 17\\
            & \equiv 2
        \end{align*}
      \item sende $B = 10$ an Alice
    \end{itemize}

    Alice:
    \begin{itemize}
      \item berechne geheimen gemeinsamen Schlüssel
        \begin{align*}
          K & \equiv B^a \mod p\\
            & \equiv 10^{10} \mod 17\\
            & \equiv 2
        \end{align*}
    \end{itemize}

\end{itemize}

\end{document}
