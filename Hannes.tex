\documentclass[
  a4paper,
  11pt,
]{article}

\usepackage[utf8]{inputenc}
\usepackage[cm, headings]{fullpage}
\usepackage[ngerman]{babel}
\usepackage{amsmath}
\usepackage{amssymb}
\renewcommand{\O}{\mathcal{O}}

\usepackage{color}
\definecolor{mygray}{rgb}{0.5,0.5,0.5}
\usepackage{listings}
\usepackage{tikz}
\usepackage{pgfplots}
\usepackage{booktabs}

% Für Zeilenumbrüche ohne Indentations
\setlength{\parindent}{0pt}

\lstset{%
  basicstyle=\ttfamily,
  keywordstyle=\color{blue},
  commentstyle=\color{mygray},
  language=C,
  showstringspaces=false,
}


% Grafikpaket laden
\usepackage{graphicx}
\usepackage{float}

% tabelle auf textbreite
\usepackage{tabularx}

% coole Kopf- und Fußzeilen:
\usepackage{fancyhdr}
% Seitenstil ist natürlich fancy:
\pagestyle{fancy}
% alle Felder löschen:
\fancyhf{}

\fancyhead[L]{%
  Seminar: Public-Key-Kryptographie
}
\fancyhead[R]{%
}
%\fancyfoot[L]{}
\fancyfoot[C]{\thepage}

\newcommand{\N}{\mathbb{N}}
\newcommand{\Z}{\mathbb{Z}}
\newcommand{\ggT}{\text{ggT}}

\title{}

\author{}

\begin{document}

\section*{Verschluesselung}
\label{sec:Verschluesselung}

Diskreter Logarithmus:
\\\\
source: Übersetzung aus HAC97 Seite 103
\\\\
Definition:
\\\\
The discrete logarithm problem (DLP) is the following:

%$Z_p^* = \{ 0 ,\ldots , p - 1 \} , {a,b\in \{0,\ldots ,p-1\},\,}$\\
%$g^{{ab}}$			%g hoch a*b
%$a\equiv b\mod m$	%kongruenz

$Z_p^* = \{ 0 ,\ldots , p - 1 \}$

given a prime p, a generator ${a\in Z_p^*}$, and an element ${b\in Z_p^*}$,
find the integer x, $0\leq x\leq p-2$, such that
a hoch x kongruent b (mod p)
\\\\
Das Diskrete-Logarithmus-Problem:
Für eine Primzahl p, eine multiplikative Gruppe $Z_p^* = \{ 0 ,\ldots , p - 1 \}$, einen Generator ${a\in Z_p^*}$ und ein ${b\in Z_p^*}$,
finde ${x\in N}$, $0\leq x\leq p-2$ (oder auch $1\leq x\le p-1$, zyklische Gruppe),
so dass a hoch %$x\equiv b\mod p$.

\subsection*{Diffie-Hellman-Merkle Schluesselaustausch}
\label{sub:Diffie-Hellman-Merkle Schluesselaustausch}

	Alice\\
	1.\\
	Wähle:	Primzahl $p$\\
	Finde:	Erzeuger $g$:\\
			$g^n\text{ }mod\text{ }p \rightarrow \{ 0 ,\ldots , p - 1 \}$\\
	Wähle:	Geheimen Schlüssel: $a\in \{ 0 ,\ldots , p - 1 \}$\\
	Berechne:	Öffentlichen Schlüssel: $A=g^a\text{ }mod\text{ }p$\\
	Sende: (p, g, A) an Bob\\\\

	Bob\\
	2.\\
	Wähle:	Geheimen Schlüssel: $b\in \{ 0 ,\ldots , p - 1 \}$\\
	Berechne:	Öffentlichen Schlüssel: $B=g^b\text{ }mod\text{ }p$\\
	Berechne:	Geheimen gemeinsamen Schlüssel: $K=A^b\text{ }mod\text{ }p$\\
	Sende: (B) an Alice\\\\
	
	Alice\\
	3.\\
	Berechne:	Geheimen gemeinsamen Schlüssel: $K=B^a\text{ }mod\text{ }p$\\
	

\begin{figure}[H]
	\centering
	\includegraphics[width=\textwidth]{Diffie-Hellman-Schluesselaustausch2.png}
	\caption{Diffie-Hellman-Merkle Schluesselaustausch}
	\label{img:grafik-dummy}
\end{figure}




\end{document}
